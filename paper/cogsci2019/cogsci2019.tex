\documentclass[]{article}
\usepackage{lmodern}
\usepackage{amssymb,amsmath}
\usepackage{ifxetex,ifluatex}
\usepackage{fixltx2e} % provides \textsubscript
\ifnum 0\ifxetex 1\fi\ifluatex 1\fi=0 % if pdftex
  \usepackage[T1]{fontenc}
  \usepackage[utf8]{inputenc}
\else % if luatex or xelatex
  \ifxetex
    \usepackage{mathspec}
  \else
    \usepackage{fontspec}
  \fi
  \defaultfontfeatures{Ligatures=TeX,Scale=MatchLowercase}
\fi
% use upquote if available, for straight quotes in verbatim environments
\IfFileExists{upquote.sty}{\usepackage{upquote}}{}
% use microtype if available
\IfFileExists{microtype.sty}{%
\usepackage{microtype}
\UseMicrotypeSet[protrusion]{basicmath} % disable protrusion for tt fonts
}{}
\usepackage[margin=1in]{geometry}
\usepackage{hyperref}
\hypersetup{unicode=true,
            pdftitle={Understanding the impacts of video activity guides on parent-child interaction},
            pdfkeywords={digital parenting advice; joint attention; lexical diversity; guided
play;},
            pdfborder={0 0 0},
            breaklinks=true}
\urlstyle{same}  % don't use monospace font for urls
\usepackage{graphicx,grffile}
\makeatletter
\def\maxwidth{\ifdim\Gin@nat@width>\linewidth\linewidth\else\Gin@nat@width\fi}
\def\maxheight{\ifdim\Gin@nat@height>\textheight\textheight\else\Gin@nat@height\fi}
\makeatother
% Scale images if necessary, so that they will not overflow the page
% margins by default, and it is still possible to overwrite the defaults
% using explicit options in \includegraphics[width, height, ...]{}
\setkeys{Gin}{width=\maxwidth,height=\maxheight,keepaspectratio}
\IfFileExists{parskip.sty}{%
\usepackage{parskip}
}{% else
\setlength{\parindent}{0pt}
\setlength{\parskip}{6pt plus 2pt minus 1pt}
}
\setlength{\emergencystretch}{3em}  % prevent overfull lines
\providecommand{\tightlist}{%
  \setlength{\itemsep}{0pt}\setlength{\parskip}{0pt}}
\setcounter{secnumdepth}{0}
% Redefines (sub)paragraphs to behave more like sections
\ifx\paragraph\undefined\else
\let\oldparagraph\paragraph
\renewcommand{\paragraph}[1]{\oldparagraph{#1}\mbox{}}
\fi
\ifx\subparagraph\undefined\else
\let\oldsubparagraph\subparagraph
\renewcommand{\subparagraph}[1]{\oldsubparagraph{#1}\mbox{}}
\fi

%%% Use protect on footnotes to avoid problems with footnotes in titles
\let\rmarkdownfootnote\footnote%
\def\footnote{\protect\rmarkdownfootnote}

%%% Change title format to be more compact
\usepackage{titling}

% Create subtitle command for use in maketitle
\providecommand{\subtitle}[1]{
  \posttitle{
    \begin{center}\large#1\end{center}
    }
}

\setlength{\droptitle}{-2em}

  \title{Understanding the impacts of video activity guides on parent-child
interaction}
    \pretitle{\vspace{\droptitle}\centering\huge}
  \posttitle{\par}
    \author{}
    \preauthor{}\postauthor{}
    \date{}
    \predate{}\postdate{}
  

\begin{document}
\maketitle
\begin{abstract}
Early parenting practices play an important role in shaping the future
outcomes of young children (Hart \& Risley, 1995; Heckman, 2006). In
particular, high-quality early interactions and language input appear to
facilitate language learning and result in higher levels of school
performance. The rise of phone- and tablet-based parenting applications
(``apps'') holds the promise of delivering low-cost, positive
interventions on parenting style to a wide variety of populations. Of
special interest are the parents of very young children, who are often
difficult to reach in other ways. Yet little is known about the effects
of communicating to parents through app-based interventions. In a study
of one commercial app offering a collection of age-appropriate activity
videos, we find that the quality of parent-child interactions increases
in some ways as a result of using the app. Specifically, the lexical
diversity of parents' child-directed speech increases, and measures of
joint attention show\ldots{}
\end{abstract}

\hypertarget{introduction}{%
\section{Introduction}\label{introduction}}

The quantity and quality of early language input has been found to be
strongly associated with later language and academic outcomes (Cartmill
et al., 2013; Hart \& Risley, 1995; Hirsh-Pasek et al., 2015; Marchman
\& Fernald, 2008). Thus, because of the potential for large downstream
effects (Heckman, 2006), there is tremendous interest in interventions
that change children's language environment. And because parents define
a large portion of that environment, especially before the onset of
formal schooling, parent behavior is a critical locus for such
interventions. Many effective parenting interventions require large
resource investments and require many hours of in-person contact
(Gertler et al., 2014; Schweinhart et al., 2004), making implementation
at scale a daunting proposition. For this reason, many researchers
targeting early language are interested in delivering parenting
interventions remotely -- through texts, apps, and videos delivered on
digital devices. But what do parents take away from these short messages
about what to do or how to talk with their children?

The content provided by digital parenting interventions runs the gamut
from general parenting messages and facts from child development
research to specific advice and suggested activities. A growing body of
evidence suggests that these digital interventions can be effective
across a range of cultures, income levels, and children's ages (for a
review, see )(Breitenstein, Gross, \& Christophersen, 2014). For
example, in contrast to a face-to-face parent training intervention, a
tablet-based version saw significantly higher session completion rates
(51\% attendance vs.~85\% module completion) and comparable or larger
effect sizes on parents' and children's (aged 2 to 5 years) behavior
(Breitenstein, Fogg, Ocampo, Acosta, \& Gross, 2016). Often, however,
the theory of change presupposed by such interventions is relatively
vague. Both within and outside the realm of academic interventions,
messages to parents of young children often seek to provide knowledge
about some aspect of development (e.g., early language), often in tandem
with a suggestion regarding activities. Such messages are assumed to
inform parents' choice of behaviors, spurring them to engage in the
target activity, which is assumed to be more stimulating than what
parents would have done otherwise.

This theory of change is grounded in ideas about guided play and early
language stimulation. Language input varies not only in quantity (i.e.,
the number of total tokens), but also in quality (i.e., the diversity of
the tokens) (Malvern, Richards, Chipere, \& Durán, 2004). Further,
language learning -- especially the acquisition of early vocabulary in
the first years -- appears to be supported preferentially by parents and
children \emph{jointly attending} to some object or activity (Bigelow,
MacLean, \& Proctor, 2004). Episodes of joint attention are frequent
during guided play, when parents set goals and scaffold their child's
activities (Weisberg, Hirsh-Pasek, \& Golinkoff, 2013; Wood, Bruner, \&
Ross, 1976). Thus, the current literature supports interventions that
encourage parents to provide high-quality language and interaction
through something like guided play -- whether via reading books or
playing with a shape-sorter at home, or via a conversation about
categories in the supermarket.

But is this theory of change correct? That is, does the provision of
knowledge and activities lead to higher-quality play? An alternative
hypothesis is that this theory could be wrong in a number of ways. By
focusing parents on a specific activity, they could over-focus on
achieving the superficial goals of the activity. This problem might be
especially likely with video messages, which could encourage parents to
try to mimic a model. And these misinterpretations in turn could result
in less high-quality talk, with less joint attention. Or, these messages
could produce the desired effect, but only for those parents who already
have a general orientation towards children's early learning.

Our current experiments were designed to make a direct test of this
question: How do parents change their interactions with young children
on the basis of short video parenting messages? In both studies, we
collected data from parent-child dyads in a local children's museum. We
showed parents in the experimental group a single short video modeling
an interactive toy-based activity along with a scientific justification.
Parents in the control group received either no video (Experiment 1) or
a video of a recent finding in developmental psychology (Experiment 2).
We then gave the toys from the video to all dyads and videotaped their
interactions, coding for language quantity and quality as well as joint
attention.

Under such expert guidance, children are encouraged and motivated to
engage in more advanced play, undertaking explorations that push the
boundaries of what they would be able to do unaided (Vygotsky, 1980).

But does providing such information change parents' behavior towards
young children? And is this advice effective in producing change along
the desired dimension? An alternative might be that asking parents to
focus on a specific activity that may be novel to them may tax their
cognitive resources, resulting in decreases in the quality of their
child-directed speech, or in less flexible, lower-quality social
interactions.

What are digital interventions With the widespread use of smartphones
and tablets worldwide, digitally-delivered interventions could address
many of the logistical barriers that have limited scaling up
face-to-face delivery methods. However, the parent and child outcomes
assessed in the review (e.g.infant positive behaviors, satisfaction,
emotional symptoms etc.) did not address the nature or quality of
parent-infant interactions at a detailed level, for example whether the
interventions lead to children paying more attention, or vocabulary
changes in parents' language usages. Although digitally-delivered
activities are designed to promote learning and cognitive development,
it is unclear how they might affect these dimensions of parent-child
interactions. Thus, we want to conduct this experiment to explore if and
how digital scaffolding of activities affect the social and linguistic
characteristics of parent-child ineractions. The quality of parent-child
interactions can be measured by both the social engagement of parents
(e.g., joint attention to objects in the environment)(Bigelow et al.,
2004) and the quality of language (e.g., vocabulary diversity) (Malvern
et al., 2004).

Communication and message content

In our study, we focus on videos that provide suggested play activities
to parents. These videos constitute a short-term ``guided play''
intervention.

Young children spend a large portion of their waking time at play,
variously manipulating objects, exploring their environment, and
interacting with caregivers and peers. Playing with objects allows them
to discover hidden object properties and relations, and to build a
causal understanding of how objects interact (e.g., Schulz \& Bonawitz,
2007). Meanwhile, play also gives children an opportunity to set and
achieve goals (e.g., build a tower) and to practice a wide range of
motor skills (e.g., stacking) that will help them navigate the world
(Singer, Golinkoff, \& Hirsh-Pasek, 2006). Social play can help children
learn about human relationships, both through imitation of adult
behaviors and by experiencing and learning to process emotional events
such as failures (Singer et al., 2006). Of course, young children are
rarely playing in isolation: caregivers often provide encouragement and
guidance while scaffolding a child's play (Kaye, 1970; Wood et al.,
1976). The quality of interactions during such guided play has been
shown to influence language learning: parents' joint attention to
objects that their child was focused on was positively correlated with
the child's subsequent vocabulary growth (M. Carpenter \& Tomasello,
1998; Tomasello \& Farrar, 1986). Episodes of joint attention during
guided play have also been found to contain more age-appropriate
advanced forms of play (Bigelow et al., 2004). More generally, parenting
practices early in childhood have been shown to play an important role
in shaping the future outcomes of young children (Hart \& Risley, 1995;
Heckman, 2006). While interventions often have trouble reaching many
parents of very young children, the proliferation of mobile devices
offers a good avenut for the digital delivery of parenting advice
(Breitenstein et al., 2014). However, the efficacy of digital parenting
advice has not been widely proven.

\hypertarget{guided-play-scaffolds-learning}{%
\subsection{Guided Play Scaffolds
Learning}\label{guided-play-scaffolds-learning}}

Children's early play behaviors are often assisted by more skilled and
knowledgeable play partners such as their caregivers and older siblings
(Kaye, 1970). Under such expert guidance, children are encouraged and
motivated to engage in more advanced play, undertaking explorations that
push the boundaries of what they would be able to do unaided (Vygotsky,
1980). These tutorial interactions have been shown to be important
components of child development (Wood et al., 1976). Thus, with the
knowledge of both play and tutorial interactions, guided play, which
consists of both active and enjoyable activities as well as close
guidance of adults (Hirsh-Pasek \& Golinkoff, 2008) has drawn
researchers' interest. A study of preschoolers showed that guided play
scaffolds the environment while still allowing children to maintain a
large degree of control, and it outperforms direct-instruction
approaches in encouraging a variety of positive academic outcomes
(Weisberg et al., 2013). Another study found that guided play could
facilitate children's vocabulary and comprehensive language development
and subsequent literacy skills (Massey, 2013).

\hypertarget{improving-parenting-practices-and-language-use}{%
\subsection{Improving Parenting Practices and Language
Use}\label{improving-parenting-practices-and-language-use}}

Thus, including guided play in parenting practices from an early age may
boost children's language and educational outcomes. A recent home-visit
parenting practices intervention targeting children of low socioeconomic
status found that parents in the intervention group gained knowledge of
language development, and that this effect sustained four months after
the intervention (Suskind et al., 2015). However, although a number of
interactive measures increased during the experiment, including the
number of word tokens, conversational turns, and child vocalizations,
these increases did not sustain after the intervention. That changes did
not sustain could be due to the intervention itself, or could merely be
that the methods of home-visiting is not sustainable enough for parents
to easily and constantly get parenting advice. Thus, new methods of
delivering parenting advice should be considered.

\hypertarget{effectiveness-of-digital-delivery}{%
\subsection{Effectiveness of Digital
Delivery}\label{effectiveness-of-digital-delivery}}

\hypertarget{experiment-1}{%
\section{Experiment 1}\label{experiment-1}}

In Experiment 1, we invited parents of 6- to 24-month-old infants
visiting the Children's Discovery Museum in San Jose to complete
activities from the Kinedu app. Parents were randomly assigned to the
video group or the control group; parents in the video group watched a
video from the Kinedu app (matched to their child's age), and then
performed the activity with their child using the props from the video.
Parents in the control group did not watch a Kinedu video, instead they
were given the same props and were told to play with their infants as
they would at home.

\hypertarget{method}{%
\subsection{Method}\label{method}}

\hypertarget{participants.}{%
\subsubsection{Participants.}\label{participants.}}

60 infants (F = 42, M = 18) aged 6-24 months (20 6-11.9 month-olds, 20
12-17.9 month-olds, and 20 18-24 month-olds) and their parents
participated in a museum in northern California. We included infants who
were exposed to English at least 50 percent of the time (n = 58) or who
were exposed less but whose participating parent reported that they
primarily speak English with their child at home (n = 2). 61\% of
participants (n = 37) had been exposed to two or more languages as
indicated by their parent. Parents identified their children as White (n
= 25), Asian (n = 11), African American/Black (n = 2), Biracial (n =
12), other (n = 5), or declined to state (n = 5). Fifteen parents
reported that their child was of Hispanic origin. Parents tended to be
highly-educated, with reports of highest level of education ranging from
completed high school (n = 5), some college (n = 6), four-year college
(n = 14), some graduate school (n = 1), to completed graduate school (n
= 28) or declined to state (n = 6).

\hypertarget{materials.}{%
\subsubsection{Materials.}\label{materials.}}

Stimuli included videos from the Kinedu (Kinedu Inc.) commercial
parenting application. The videos were designed to show activities to
parents that they could perform with their child in order to foster
cognitive and physical development, and were targeted to the child's age
and level of development. In each video, an adult and child perform the
activity while a narrator explains the activity and its purpose. We
selected two videos for each of three age groups in our sample (6-11.9
months, 12-17.9 months, 18-23.94 months). More information about the
specific videos is available in the Appendix. Participants were also
given a set of toys corresponding to those in the video that they
watched so that they could complete the activity. The toys associated
with each video are listed in the Appendix.

Participants were randomly assigned to either the \emph{Video} condition
or the \emph{Control} condition. Parents participating in the Video
condition were assigned to watch one of the two activity videos
available for their child's age group, while parents in the Control
condition watched no activity video, and were simply asked to play with
their child as they normally would. The Control condition was yoked to
the Activity Video condition such that for every participant in the
Video condition who saw a particular video and received the associated
props, a participant in the Control condition received the same props
but did not watch the activity video. Parents also completed the Early
Parenting Attitudes Questionnaire (EPAQ; (Hembacher \& Frank, 2018)).
The EPAQ measures parents' attitudes about parenting and child
development along three dimensions: rules and respect, early learning,
and affection and attachment.

\hypertarget{procedure.}{%
\subsubsection{Procedure.}\label{procedure.}}

After providing informed consent, parents in the Video condition watched
the assigned activity video on a laptop with headphones. To ensure that
parents could give the video their full attention, the experimenter
played with the infant with a set of toys (different from the
experimental props used in the study) while the video was being played.
Immediately following the video, each parent-child dyad was provided
with the props to complete the activity the parent had viewed. The toys
were placed on a large foam mat, and parents were instructed to sit on
the mat with their child and re-create the activity they had viewed for
a period of three minutes. In the Control condition, after consenting
parents were told to play with their child as they would at home with
the provided props for a period of three minutes. They were not given
any additional instructions about how to use the props.

In both conditions, two video cameras were used to record the play
session from different angles, and parents were fitted with a wireless
microphone (Shurre lavalier microphone) to record their child-directed
speech. After three minutes of play had elapsed, parents were told they
could stop playing and the cameras and microphone were turned off.
Parents were then asked to complete the EPAQ before being debriefed.

\hypertarget{joint-attention-coding-procedure.}{%
\subsubsection{Joint Attention Coding
Procedure.}\label{joint-attention-coding-procedure.}}

The video of each session was manually coded for episodes of joint
attention (JA) using the Datavyu software (Team, 2014). The video taken
at floor level was coded by default, but the other video was referred to
if the participants were not visible or if there was technical
difficulty with the first camera. Each session's video was coded for
episodes of coordinated JA, episodes of passive JA, and parental bids
for JA. Parental bids for JA were defined as any attempt to initiate
joint attention (i.e labeling, pointing, or otherwise drawing attention
to an object) that did not result in passive or coordinated JA. If more
than 3 seconds elapsed between bids, they were coded as separate
attempts. An episode of joint attention was considered passive if both
participants visually focused on an object for a minimum of 3 seconds
but the child did not acknowledge the parent. If either participant
looked away from the object for less than 3 seconds and then returned to
the same object it was considered part of the same period of joint
attention. A joint attention episode was considered coordinated if both
participants visually focused on an object for a minimum of 3 seconds
and at some point in the interaction the child indicated awareness of
interaction with some overt behavior toward the parent such as looking
at their face, gesturing, vocalizing, or turn-taking.

A second coder independently coded a third of the videos (i.e., 20 of
the 60 videos, approximately equally distributed across ages) to
establish reliability. The two coders had a reliability of ICC = 0.80
with 95\% confident interval (CI) = {[}0.57,0.92{]} (p \textless{} 0.05)
for number of parent bids for JA; ICC = 0.20 with 95\% CI =
{[}-0.26,0.58{]} for number of passive JA episodes; ICC = 0.66 with 95\%
CI = {[}0.32,0.85{]} (p \textless{} 0.05) for number of coordinated JA
episodes; ICC = 0.24 with 95\% CI = {[}-0.21,0.61{]} for total duration
of passive JA episodes, and ICC = 0.62 with 95\% CI = {[}0.27,0.83{]} (p
\textless{} 0.05) for total duration of coordinated JA episodes.

\hypertarget{results}{%
\subsection{Results}\label{results}}

The transcripts and hand-coded behavioral data was analyzed according to
our preregistration\footnote{Preregistration:
  \url{https://osf.io/2bpdf/}{]}}. Below we first describe the lexical
diversity results, followed by the joint attention results.

\hypertarget{lexical-diversity}{%
\subsubsection{Lexical Diversity}\label{lexical-diversity}}

Parents' child-directed speech during the play sessions was transcribed.
For each transcript, the words were lemmatized using Honnibal (2017),
and the word \emph{types} (unique words) and \emph{tokens} (total words)
were then tallied and the type-token ratio (TTR) calculated as a measure
of lexical diversity. Although TTR was our preregistered measure of
lexical diversity, TTR is correlated with the length of a text, whereas
the measure of textual lexical diversity (MTLD) is not (McCarthy \&
Jarvis, 2010). Thus, we also measure lexical diversity with MTLD, which
is calculated as the mean length of sequential word strings in a text
that maintain a given TTR value (here, .720).

We fit a mixed-effects linear regression predicting TTR as a function of
condition, age (scaled and 0-centered), gender, and parent's education
level with a random intercept per video using lme4 (Bates, Mächler,
Bolker, \& Walker, 2015). There was significantly lower TTR in the Video
condition (mean: 0.32) than in the Control condition (mean: 0.43,
\(\beta=-.12\), t(39.9) = 4.25, \emph{p}\textless{}.001). There were no
significant effects of age, gender, or parent's level of education. A
similar mixed-effects linear regression instead predicting MTLD also
found significantly lower lexical diversity in the Video condition
(mean: 15.7) than in the Control condition (mean: 21.9,
\(\beta=-11.67\), t(43) = 3.08, \emph{p}\textless{}.01), with no other
significant effects. Figure 1 shows the mean of each lexical diversity
measure (TTR and MTLD) by condition.

We also conducted similar regressions predicting the number of word
tokens and types, finding only a significant effect of condition on the
number of word tokens (\(\beta=57.23\), t(34.4)=2.19,
\emph{p}\textless{}.05), with parents using more words in the Video
condition (mean: 225, 95\% CI: {[}197,252{]}) than in the Control
condition (mean: 165, 95\% CI: {[}139,193{]}).

(Table with mean and SD of tokens, types, and TTR)

\begin{figure}[H]

{\centering \includegraphics{figs/e1lex_div-1} 

}

\caption{Mean lexical diversity scores by condition (left: Type/Token ratio, right: MTLD) in Experiment 1. Error bars show bootstrapped 95 percent confidence intervals (CIs).}\label{fig:e1lex_div}
\end{figure}

Word tokens and word types

\begin{figure}[H]

{\centering \includegraphics{figs/e1token_type-1} 

}

\caption{Mean number of word types and word tokens by condition in Experiment 1.}\label{fig:e1token_type}
\end{figure}

\hypertarget{joint-attention}{%
\subsubsection{Joint Attention}\label{joint-attention}}

We fit a mixed-effects linear regression predicting the number of bids
for joint attention (JA) as a function of fixed effects of condition,
age (scaled and 0-centered), gender, parent's education level, and the
subscales of the EPAQ: Early Learning (EL), Affection and Attachment
(AA), and Rules and Respect (RR), along with interactions of condition
and EL, AA, and RR. This lme4 model included random intercepts per
video. There were significantly more bids for JA in the Video condition
(mean: 6.24, sd: 2.79) than in the Control condition (mean: 3.56, sd:
2.50, \(\beta=3.51\), t(40.3) = 2.95, \emph{p}\textless{}.01). There
were no other significant effects. Mixed-effects regressions with the
same structure were performed predicting the number of episodes of
coordinated and passive JA, and the total duration of time spent in
coordinated and passive JA. There were no significant effects on the
number or total duration of coordinated JA episodes, nor on the total
duration of passive JA episodes. For the regression predicting the
number of passive JA episodes, the only significant effect was an
interaction of condition and RR (\(\beta=1.83\), t(41.5) = 2.22,
\emph{p}\textless{}.05), showing that for parents in the Video
condition, those with higher Rules and Respect subscores engaged in more
passive JA episodes. Figure X shows \ldots{} by condition.

\begin{figure}[H]

{\centering \includegraphics{figs/e1ja-graphs-1} 

}

\caption{Mean number of bids and episodes of Joint Attention in Experiment 1.}\label{fig:e1ja-graphs}
\end{figure}

\begin{figure}[H]

{\centering \includegraphics{figs/e1ja-graphs-pass-coord-1} 

}

\caption{Average number of passive and coordinated episodes of JA in Experiment 1.}\label{fig:e1ja-graphs-pass-coord}
\end{figure}

\hypertarget{discussion}{%
\subsection{Discussion}\label{discussion}}

In summary, while parents produced more word types and tokens after
viewing the activity video, lexical diversity (Type/Token ratio and
MTLD) was higher when parents were just asked to play as they normally
would. This may suggest that parents in the Video condition are being
more repetitive in their attempt to stick to the task prescribed in the
video. Demographics and EPAQ do not interact with condition, but there
is a marginal effect of RR score on lexical diversity (lower diversity
for higher RR scores), and marginal effects of parent education on word
types and tokens (more types and tokens for higher parent education).

There was a main effect of condition on total bids for joint attention.
Parents in the Video condition, after seeing a video demonstrating an
activity, made a greater number of bids for joint attention with their
child. There was no effect of condition on the number of episodes of
either passive or coordinated joint attention, or the duration of these
episodes. There was a marginal effect of gender on bids for joint
attention, with parents of males producing more bids. There was a
marginal interaction between RR scores and condition on passive joint
attention, such that the experimental condition increased the number of
episodes of PJA to a greater extent for people with high RR scores.
While the electronically-delivered parenting advice increased the number
of bids for joint attention by parents, it did not significantly effect
the number or duration of episodes of joint attention. It may be that
child variables had a larger relative impact on the attainment of joint
attention.

\hypertarget{experiment-2}{%
\section{Experiment 2}\label{experiment-2}}

Experiment 1 found that parents who watched a Kinedu video spoke more
words overall, but had lower lexical diversity compared to parents who
played with their children as they normally would at home. Parents who
watched a Kinedu video also made more bids for joint attention, although
these bids did not result in more episodes of joint attention compared
to the control group. Experiment 2 attempts to replicate these findings
from with a restricted number of preregistered predictions
(\href{https://osf.io/2bpdf/}{link}).

\hypertarget{method-1}{%
\subsection{Method}\label{method-1}}

\hypertarget{participants.-1}{%
\subsubsection{Participants.}\label{participants.-1}}

83 infants (F = 37, M = 47) aged 12-24 months (42 12-17.9 month-olds, 42
18-24 month-olds) and their parents participated in the same museum as
Experiment 1. We included infants who were exposed to English at least
75 percent of the time or who were exposed less but whose participating
parent reported that they primarily speak English with their child at
home. Forty nine\% of participants (n = 41) had been exposed to two or
more languages as indicated by their parent. Parents identified their
children as White (n = 39), Asian (n = 20), African American/Black (n =
1), Biracial (n = 9), other (n = 7), or declined to state (n = 8).
Sixteen parents reported their child was of Hispanic origin. Parents
tended to be highly- educated, with reports of highest level of
education ranging from completed high school (n = 0), some college (n =
5), four-year college (n = 28), some graduate school (n = 2), to
completed graduate school (n = 35) or declined to state (n = 14).

\hypertarget{materials.-1}{%
\subsubsection{Materials.}\label{materials.-1}}

The design of Experiment 2 was similar to that of Experiment 1, except
that instead of a No-Video control condition, parents instead watched a
video that was generally related to child development research, but did
not give any specific instructions about how to interact with infants or
children. This was to control for the possibility that differences in
language output and joint attention in Experiment 1 could be due to
simply cuing parents to think about infants' learning and cognitive
development. The videos presented in the Control Video condition were
media clips (available on YouTube) of developmental psychologists
explaining their research interleaved with footage of infants or
toddlers engaged in developmental research studies. Thus, the content of
the videos superficially matched those in the Activity Video condition,
but did not suggest any particular activities. The videos were trimmed
to approximately match the average video length in the Activity Video
condition (close to 90 s).

\hypertarget{procedure.-1}{%
\subsubsection{Procedure.}\label{procedure.-1}}

The procedure for Experiment 2 matched that of Experiment 1, except that
parents in the Control Video condition watched a control video before
the play session. Consistent with the No-Video control condition in
Experiment 1, parents in the Control Video condition were told to play
with their child as they would at home, and were not given additional
instructions. The coding procedure also matched that of Experiment 1. A
second coder independently coded a third of the videos (i.e., 26 of the
84 videos, approximately equally distributed across ages) to establish
reliability. The two coders had a reliability of ICC = 0.80 with 95\%
confident interval (CI) = {[}0.60,0.90{]} (p \textless{} 0.05) for
number of parent bids for JA; ICC = 0.74 with 95\% CI = {[}0.59,0.87{]}
(p \textless{} 0.05) for number of passive JA episodes; ICC = 0.78 with
95\% CI = {[}0.58,0.90{]} (p \textless{} 0.05) for number of coordinated
JA episodes; ICC = 0.72 with 95\% CI = {[}0.46,0.86{]} (p \textless{}
0.05) for total duration of passive JA episodes, and ICC = 0.88 with
95\% CI = {[}0.75,0.94{]} (p \textless{} 0.05) for total duration of
coordinated JA episodes.

\hypertarget{results-1}{%
\subsection{Results}\label{results-1}}

Parents' child-directed speech was transcribed and processed according
to the same procedure used in Experiment 1.

\hypertarget{lexical-diversity-1}{%
\subsubsection{Lexical Diversity}\label{lexical-diversity-1}}

We fit a mixed-effects linear regression predicting TTR and MTLD as a
function of age (scaled and 0-centered) and condition with an
interaction term, and with random intercepts per video using lme4 (Bates
et al., 2015). There was significantly lower TTR in the Video condition
(mean: 0.38) than in the Control condition (mean: 0.47, \(\beta=-.09\),
t(8.7) = 3.06, \emph{p}=.01). There was no significant effect of age. A
similar mixed-effects linear regression instead predicting MTLD found no
significant effects of age or condition. Figure X shows the mean of each
lexical diversity measure (TTR and MTLD) by condition. Regressions with
the same structure predicting the number of word tokens and types found
no significant effects of age or condition.

(Table with mean and SD of tokens, types, TTR, MTLD)

\begin{figure}[H]

{\centering \includegraphics{figs/e2lexdiv-1} 

}

\caption{Mean lexical diversity scores by condition (left: Type/Token ratio, right: MTLD) in Experiment 2.}\label{fig:e2lexdiv}
\end{figure}

\begin{figure}[H]

{\centering \includegraphics{figs/e2token-type-1} 

}

\caption{Mean number of word types and word tokens by condition in Experiment 2.}\label{fig:e2token-type}
\end{figure}

\hypertarget{joint-attention-1}{%
\subsubsection{Joint Attention}\label{joint-attention-1}}

\begin{figure}[H]

{\centering \includegraphics{figs/e2ja-graphs-1} 

}

\caption{Mean number of bids and episodes of Joint Attention in Experiment 2.}\label{fig:e2ja-graphs}
\end{figure}

\begin{figure}[H]

{\centering \includegraphics{figs/e2ja-graphs-pass-coord-1} 

}

\caption{Average number of passive and coordinated episodes of JA in Experiment 2.}\label{fig:e2ja-graphs-pass-coord}
\end{figure}

There was a marginal effect of condition on total bids for joint
attention. Parents in the experimental condition (i.e., those who saw a
video demonstrating an activity) made a greater number of bids for joint
attention with their child. There was no effect of condition on the
number of episodes of passive or coordinated JA, nor on the duration of
these episodes. For passive joint attention, there was a main effect of
age on both the number and duration of episodes, with older children
having fewer episodes of passive JA and episodes of shorter duration.
Moreover, older children had longer duration episodes of coordinated JA.
These results suggest that as children age, they become more socially
engaged in interactions with their caregivers.

\hypertarget{discussion-1}{%
\section{Discussion}\label{discussion-1}}

In both experiments, the number of tokens was higher in the experimental
condition, while the number of types and lexical diversity (Type/Token
ratio) were higher in the control condition. Parents may be relatively
more repetetive in the experimental condition since they are attempting
to stick to a specific prescribed task, but they talk more overall.

\hypertarget{acknowledgements}{%
\section{Acknowledgements}\label{acknowledgements}}

This work was supported by a gift from Kinedu, Inc. Thanks to members of
the Language and Cognition Lab at Stanford for helpful discussion.

\hypertarget{references}{%
\section{References}\label{references}}

\setlength{\parindent}{-0.1in} 
\setlength{\leftskip}{0.125in}

\noindent

\hypertarget{appendix}{%
\section{Appendix}\label{appendix}}

\hypertarget{experiment-1-1}{%
\subsection{Experiment 1}\label{experiment-1-1}}

\hypertarget{video-a-6-11.9-months-pick-it-up}{%
\subsubsection{Video A (6-11.9 months) ``Pick it
up''}\label{video-a-6-11.9-months-pick-it-up}}

Parents are told to encourage their child to pick up and drop individual
objects. They are also encouraged to place toys on a small cloth and
show the child that they can drag the cloth towards them to reach the
toys.

Props: cloth, plastic horse, plastic sheep, plastic elephant, toy car

\hypertarget{video-b-6-11.9-months-animal-sounds}{%
\subsubsection{Video B (6-11.9 months) ``Animal
sounds''}\label{video-b-6-11.9-months-animal-sounds}}

Parents are told to call different animals and imitate different sounds
the animals make. They are also encourgaed to observe which animal the
child prefers.

Props: plastic sheep, plastic horse, plastic frog, plastic cow, bowls

\hypertarget{video-c-12-17.9-months-give-me-the-toy}{%
\subsubsection{Video C (12-17.9 months) ``Give me the
toy''}\label{video-c-12-17.9-months-give-me-the-toy}}

Parents are told to ask their child to hand over individual toys. They
are also encouraged to praise the child after they give them the toys,
and repeat the process until the child could follow the verbal
instructions.

Props: toy boat, plastic frog, plastic elephant, toy bus

\hypertarget{video-d-12-17.9-months-classifying-my-toys}{%
\subsubsection{Video D (12-17.9 months) ``Classifying my
toys''}\label{video-d-12-17.9-months-classifying-my-toys}}

Parents are told to place toys of different sizes (big or small) in two
hoops. They are also encouraged to ask their child to distinguish
between two objects and identify which one is larger.

Props: two yellow and green rings, big car, small car, big horse, small
horse

\hypertarget{video-e-18-23.9-months-my-toys}{%
\subsubsection{Video E (18-23.9 months) ``My
toys''}\label{video-e-18-23.9-months-my-toys}}

Parents are told to show the child toys of the same shape but different
sizes, to place one of the objects in a basket and to ask the child to
take out the object. They are also encouraged to ask their child if the
object is bigger or smaller compared to its pair.

Props: two buckets, big car, small car, big horse, small horse

\hypertarget{video-f-18-23.9-months-the-orchestra}{%
\subsubsection{Video F (18-23.9 months) ``The
Orchestra''}\label{video-f-18-23.9-months-the-orchestra}}

Parents are told to give their child a musical instrument to play. They
are also encouraged to play a song and see if the child follows the
rhythm.

Props: maracas, drum, tambourine, clapper

\hypertarget{experiment-2-1}{%
\subsection{Experiment 2}\label{experiment-2-1}}

\hypertarget{video-a-12-17.9-months-give-me-the-toy}{%
\subsubsection{Video A (12-17.9 months) ``Give me the
toy''}\label{video-a-12-17.9-months-give-me-the-toy}}

Parents are told to ask their child to hand over individual toys. They
are also encouraged to praise the child after they give them the toys,
and repeat the process until the child could follow the verbal
instructions.

Props: plastic pig, plastic horse, plastic dog, plastic cat, plastic cow

\hypertarget{video-b-12-17.9-months-classifying-my-toys}{%
\subsubsection{Video B (12-17.9 months) ``Classifying my
toys''}\label{video-b-12-17.9-months-classifying-my-toys}}

Parents are told to place toys of different sizes (big or small) in two
hoops. They are also encouraged to ask their child to distinguish
between two objects and identify which one is larger.

Props: two yellow and green rings, big car, small car, big horse, small
horse

\hypertarget{video-c-12-17.9-months-geometric-shapes-jigzsaw-puzzle}{%
\subsubsection{Video C (12-17.9 months) ``Geometric shapes jigzsaw
puzzle''}\label{video-c-12-17.9-months-geometric-shapes-jigzsaw-puzzle}}

Parents are told to encourage their child to name different shapes on a
jigzsaw puzzle. Then they are told to undo the puzzle and invite the
child to complete the puzzle.

Props: A jigzsaw puzzle of geometric shapes

\hypertarget{video-d-18-23.9-months-my-toys}{%
\subsubsection{Video D (18-23.9 months) ``My
toys''}\label{video-d-18-23.9-months-my-toys}}

Parents are told to show the child toys of the same shape but different
sizes, to place one of the objects in a basket and to ask the child to
take out the object. They are also encouraged to ask their child if the
object is bigger or smaller compared to its pair.

Props: two buckets, big car, small car, big horse, small horse

\hypertarget{video-e-18-23.9-months-the-orchestra}{%
\subsubsection{Video E (18-23.9 months) ``The
Orchestra''}\label{video-e-18-23.9-months-the-orchestra}}

Parents are told to give their child a musical instrument to play. They
are also encouraged to play a song and see if the child follows the
rhythm.

Props: maracas, drum, tambourine, clapper

\hypertarget{video-f-18-23.9-months-my-yellow-toys}{%
\subsubsection{Video F (18-23.9 months) ``My Yellow
Toys''}\label{video-f-18-23.9-months-my-yellow-toys}}

Parents are told to show their child yellow toys and to ask ``what color
are they''. They are also told to give the child toys of different
colors, ask them to only play with the yellow ones and praise the child
after they do so.

Props: blue car, yellow car, yellow block, red block, blue block, green
block

\hypertarget{refs}{}
\leavevmode\hypertarget{ref-lme4}{}%
Bates, D., Mächler, M., Bolker, B., \& Walker, S. (2015). Fitting linear
mixed-effects models using lme4. \emph{Journal of Statistical Software},
\emph{67}(1), 1--48. \url{http://doi.org/10.18637/jss.v067.i01}

\leavevmode\hypertarget{ref-Bigelow2004}{}%
Bigelow, A. E., MacLean, K., \& Proctor, J. (2004). The role of joint
attention in the development of infants' play with objects.
\emph{Developmental Science}, \emph{7}, 518--526.

\leavevmode\hypertarget{ref-Breitenstein2016}{}%
Breitenstein, S. M., Fogg, L., Ocampo, E. V., Acosta, D. I., \& Gross,
D. (2016). Parent use and efficacy of a self-administered, tablet-based
parent training intervention: A randomized controlled trial. \emph{JMIR
mHealth and uHealth}, \emph{4}(2), e36.
\url{http://doi.org/10.2196/mhealth.5202}

\leavevmode\hypertarget{ref-Breitenstein2014}{}%
Breitenstein, S. M., Gross, D., \& Christophersen, R. (2014). Digital
delivery methods of parenting training interventions: A systematic
review. \emph{Worldviews on Evidence-Based Nursing}, \emph{11},
168--176.

\leavevmode\hypertarget{ref-Cartmill2013}{}%
Cartmill, E. A., Armstrong, B. F., Gleitman, L. R., Goldin-Meadow, S.,
Medina, T. N., \& Trueswell, J. C. (2013). Quality of early parent input
predicts child vocabulary 3 years later. \emph{Proceedings of the
National Academy of Sciences}, \emph{110}(28), 11278--11283.
\url{http://doi.org/10.1073/pnas.1309518110}

\leavevmode\hypertarget{ref-Jamaica2014}{}%
Gertler, P., Heckman, J., Pinto, R., Zanolini, A., Vermeersch, C.,
Walker, S., \ldots{} Grantham-McGregor, S. (2014). Labor market returns
to an early childhood stimulation intervention in jamaica.
\emph{Science}, \emph{344}(6187), 998--1001.
\url{http://doi.org/10.1126/science.1251178}

\leavevmode\hypertarget{ref-Hart1995}{}%
Hart, B., \& Risley, T. R. (1995). \emph{Meaningful differences in the
everyday experience of young american children}. Baltimore, MD: Brookes.

\leavevmode\hypertarget{ref-Heckman2006}{}%
Heckman, J. J. (2006). Skill formation and the economics of investing in
disadvantaged children. \emph{Science}, \emph{312}(5782), 1900--1902.
\url{http://doi.org/10.1126/science.1128898}

\leavevmode\hypertarget{ref-Hembacher2018}{}%
Hembacher, E., \& Frank, M. C. (2018). The early parenting attitudes
questionnaire: Measuring intuitive theories of parenting and child
development. \emph{PsyArXiv}. \url{http://doi.org/10.31234/osf.io/hxk3d}

\leavevmode\hypertarget{ref-HirshPasek2015}{}%
Hirsh-Pasek, K., Adamson, L. B., Bakeman, R., Owen, M. T., Golinkoff, R.
M., Pace, A., \ldots{} Suma, K. (2015). The contribution of early
communication quality to low-income children's language success.
\emph{Psychological Science}, \emph{26}(7), 1071--1083.
\url{http://doi.org/10.1177/0956797615581493}

\leavevmode\hypertarget{ref-Hirsh2008}{}%
Hirsh-Pasek, K., \& Golinkoff, R. M. (2008). Why play= learning.
\emph{Encyclopedia on Early Childhood Development}, 1--7.

\leavevmode\hypertarget{ref-spacy2}{}%
Honnibal, I., Matthew AND Montani. (2017). SpaCy 2: Natural language
understanding with bloom embeddings, convolutional neural networks and
incremental parsing. \emph{To Appear}.

\leavevmode\hypertarget{ref-Kaye1970}{}%
Kaye, K. (1970). Mother-child instructional interaction.
\emph{Unpublished Doctoral Thesis, Department of Psychology, Harvard
University}.

\leavevmode\hypertarget{ref-Malvern2004}{}%
Malvern, D., Richards, B. J., Chipere, N., \& Durán, P. (2004).
\emph{Lexical diversity and language development}. Palgrave Macmillan.

\leavevmode\hypertarget{ref-Marchman2008}{}%
Marchman, V. A., \& Fernald, A. (2008). Speed of word recognition and
vocabulary knowledge in infancy predict cognitive and language outcomes
in later childhood. \emph{Developmental Science}, \emph{11}, F9--F16.

\leavevmode\hypertarget{ref-Massey2013}{}%
Massey, S. L. (2013). From the reading rug to the play center: Enhancing
vocabulary and comprehensive language skills by connecting storybook
reading and guided play. \emph{Early Childhood Education Journal},
\emph{41}, 125--131.

\leavevmode\hypertarget{ref-Carpenter1998}{}%
M. Carpenter, K. Nagell, \& Tomasello, M. (1998). Social cognition,
joint attention, and communicative competence from 9 to 15 months of
age. \emph{Monographs of the Society for Research in Child Development},
\emph{63}(4), i--vi, 1--143.

\leavevmode\hypertarget{ref-McCarthy2010}{}%
McCarthy, P. M., \& Jarvis, S. (2010). MTLD, vocd-d, and hd-d: A
validation study of sophisticated approaches to lexical diversity
assessment. \emph{Behavior Research Methods}, \emph{42}(2), 381--392.

\leavevmode\hypertarget{ref-Schulz2007}{}%
Schulz, L., \& Bonawitz, E. (2007). Serious fun: Preschoolers engage in
more exploratory play when evidence is confounded. \emph{Developmental
Psychology}, \emph{43}(4), 1045--1050.

\leavevmode\hypertarget{ref-PerryPreschool2004}{}%
Schweinhart, L. J., Montie, J., Xiang, Z., Barnett, W. S., Belfield, C.
R., \& Nores, M. (2004). \emph{Lifetime effects: The highscope perry
preschool study through age 40}. Ypsilanti, MI: HighScope Press.

\leavevmode\hypertarget{ref-Singer2006}{}%
Singer, D. G., Golinkoff, R. M., \& Hirsh-Pasek, K. (Eds.). (2006).
\emph{Play = learning: How play motivates and enhances children's
cognitive and social-emotional growth}. New York, NY: Oxford University
Press.

\leavevmode\hypertarget{ref-Suskind2015}{}%
Suskind, D. L., Leffel, K. R., Graf, E., Hernandez, M. W., Gunderson, E.
A., Sapolich, S. G., \ldots{} Levine, S. C. (2015). A parent-directed
language intervention for children of low socioeconomic status: A
randomized controlled pilot study. \emph{Journal of Child Language}.
\url{http://doi.org/10.1017/S0305000915000033}

\leavevmode\hypertarget{ref-datavyu}{}%
Team, D. (2014). Datavyu: A video coding tool. \emph{Databrary Project}.
Retrieved from \url{http://datavyu.org}

\leavevmode\hypertarget{ref-Tomasello1986}{}%
Tomasello, M., \& Farrar, M. J. (1986). Joint attention and early
language. \emph{Child Development}, \emph{57}(6), 1454--1463.

\leavevmode\hypertarget{ref-Vygotsky1980}{}%
Vygotsky, L. S. (1980). Mind in society: The development of higher
psychological processes.

\leavevmode\hypertarget{ref-Weisberg2013}{}%
Weisberg, D. S., Hirsh-Pasek, K., \& Golinkoff, R. M. (2013). Guided
play: Where curricular goals meet a playful pedagogy. \emph{Mind, Brain,
and Education}, \emph{7}, 104--112.

\leavevmode\hypertarget{ref-Wood1976}{}%
Wood, D., Bruner, J. S., \& Ross, G. (1976). The role of tutoring in
problem solving. \emph{Journal of Child Psychology and Psychiatry},
\emph{17}, 89--100.


\end{document}
